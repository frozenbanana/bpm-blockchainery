% This is samplepaper.tex, a sample chapter demonstrating the
% LLNCS macro package for Springer Computer Science proceedings;
% Version 2.20 of 2017/10/04
% TEMPLATE: https://www.springer.com/de/it-informatik/lncs/conference-proceedings-guidelines
%
\documentclass[runningheads]{llncs}

\usepackage{graphicx}
\usepackage{todonotes}
\newcommand{\comment}[1]{}

% Used for displaying a sample figure. If possible, figure files should
% be included in EPS format.
%\usepackage{comment}
\newcommand{\comment}[1]{}
%\usepackage{lettrine} % if you want the first letter to be \letterine{b}{igger}
% If you use the hyperref package, please uncomment the following line
% to display URLs in blue roman font according to Springer's eBook style:
% \renewcommand\UrlFont{\color{blue}\rmfamily}

%\newcommand{\rom}[1]{\uppercase\expandafter{\italic\romannumeral #1\relax}}  % roman numbers with \rom{num}
%\newcommand{\ber}[1]{\lowercase\expandafter{\italic #1\relax}} % a
\newcommand{\rom}[1]{\textit{#1}}
\newcommand{\ber}[1]{\textit{#1}}

\newcommand{\reffig}[1]{Figure~\ref{#1}}
\newcommand{\reftab}[1]{Table~\ref{#1}}
\newcommand{\refsec}[1]{Section~\ref{#1}}



\begin{document}
%
\title{Towards private Active Choreographies on public blockchain}
%
%\titlerunning{Abbreviated paper title}
% If the paper title is too long for the running head, you can set
% an abbreviated paper title here
%
\author{Henry Bergstroem\inst{1} \and
Jan Mensch\inst{1, 2}}
%
\authorrunning{F. Author et al.}
% First names are abbreviated in the running head.
% If there are more than two authors, 'et al.' is used.
%

\institute{Hasso-Plattner-Institut, Prof.-Dr.-Helmert-Straße 2-3, 14482 Potsdam, Germany 
\email{	bergstroem@uni-potsdam.de}
\and
University of Potsdam, Am Neuen Palais 10, 14469 Potsdam, Germany  \\
\email{jan.mensch@uni-potsdam.de}}

%
\maketitle              % typeset the header of the contribution
%
\begin{abstract}

    Make it clear that we are talking about an implementation of ACs, not the BPMN standard!
    
    \todo{write abstract}
    
    We are working towards implementing a realization of ACs, that implements some visibility constraints that you would usually see in real-world applications, like confidential data and transactions.

\end{abstract}

\section{Introduction} \label{sec:intro}

privacy in inter-organizational process execution.

The goal of this paper is to explore the notions of privacy and visibility with regards to untrusted execution of choreographies on the blockchain.



\section{Background and Motivation} \label{sec:backgrounmotivation}

\subsection{Blockchain-Based choreographies} \label{subsec:blockchainbased}

basically a more specific Introduction

what did Ingo do?

what did Jan do?




\subsection{Privacy Considerations} \label{subsec:privacy}

However, while discussing privacy issues, Weber et al. and Ladleif did not take any measures to actually implement them...




\textbf{Visibility levels}. \cite{ladleif}
\comment{Our aim is to restrict the visibility of ACs and thus provide privacy for the participating entities in an untrusted business process. To structure our approach we would like to subdivide the visibility of ACs, an approach introduced by Ladleif in \cite{ladleif}.}

\begin{itemize}
    \item \ber{Model level}: The logic of the business process. If you would model a business process using BPMN, the BPMN model would be the model layer. If you would implement it using a programming language, the source code would be the model layer.
    \item \ber{Communication level}: The knowledge of which messages were exchanged, who exchanged them and which entities are part of the business process. Latleif also includes the time at which messages were send. For the scope of this publication, we would like to exclude the time of an exchanged message from the communication level.
    \item \ber{Content level}: The content of the exchanged messages.
\end{itemize}



talk about layers. Why are they important? 

Motivation!





\subsection{Privacy Enhancing Technologies} \label{subsec:technologies}

talk about technologies that are currently out there 

try to be a bit more high-level, not too focused on possible solutions










\section{Approach} \label{sec:approach}

\subsection{Assumptions and Scenario} \label{subsec:assumptions}




Before introducing our schema, we would like to mention the assumptions on which our proposal is based on. Each premise is marked for potential later reference. \todo{do I need this? Am I referring to any of the specific points later on?}



\bigbreak
\textbf{Assumptions about the participating parties}: \ber{a} There are several parties which want to collaborate. These parties have \ber{b} neither trust in each other nor \ber{c} trust in any third party. They furthermore \ber{d} want to keep their business processes secret, \ber{e} hide with whom they are collaborating and \ber{f} hide what messages are exchanged during the collaboration. Since the parties distrust each other they also \ber{g} only want to share messages with the entities that have to see them and keep them secret from the others. Each party \ber{f} is "honest but curious" \todo{cite a reference here}. They follow the protocol, but will try to collect as much information as possible. These assumptions justify that we aim to keep all levels mentioned in \refsec{subsec:privacy} \todo{when refering to a subsection, do I have to write subsection x.y?} private. To be more specific, \ber{d} requires a private \ber{Model Level}, \ber{e} a private \ber{Communication Level} and \ber{f} a private  \ber{Content Level}.


\bigbreak
\textbf{Further assumptions:}
All data that is \ber{g} processed on the blockchain is considered public. It is \ber{h} possible to convert a business process into a program or state-machine in order for parties to determine that validity of a state change.




\subsection{Proposed Schema} \label{subsec:schema}

our schema


\section{Evaluation} \label{sec:eval}

implementation


\section{Discussion} \label{sec:discussion}

\begin{itemize}
    \item shortcomings
    \item How could mentioned in \refsec{subsec:technologies} improve \refsec{sec:eval}?
    \item How did it work out in the end? 
    \item private blockchain
    \item Parity
    \item future work
\end{itemize}


\section{Conclusion} \label{sec:conclusion}

final words




\bibliographystyle{splncs04} % BibTeX users should specify bibliography style 'splncs04'.
\bibliography{refs} % Entries are in the "refs.bib" file


\end{document}
